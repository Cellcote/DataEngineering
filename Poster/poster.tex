%%%%%%%%%%%%%%%%%%%%%%%%%%%%%%%%%%%%%%%%%
% baposter Portrait Poster
% LaTeX Template
% Version 1.0 (15/5/13)
%
% Created by:
% Brian Amberg (baposter@brian-amberg.de)
%
% This template has been downloaded from:
% http://www.LaTeXTemplates.com
%
% License:
% CC BY-NC-SA 3.0 (http://creativecommons.org/licenses/by-nc-sa/3.0/)
%
%%%%%%%%%%%%%%%%%%%%%%%%%%%%%%%%%%%%%%%%%

%----------------------------------------------------------------------------------------
%	PACKAGES AND OTHER DOCUMENT CONFIGURATIONS
%----------------------------------------------------------------------------------------

\documentclass[archE1,portrait,draft]{baposter}

%841mm x 1189mm
\usepackage[font=small,labelfont=bf]{caption} % Required for specifying captions to tables and figures
\usepackage{booktabs} % Horizontal rules in tables
\usepackage{relsize} % Used for making text smaller in some places
\usepackage[urlcolor  = blue]{hyperref}
\usepackage{float}
\graphicspath{{figures/}} % Directory in which figures are stored

\definecolor{bordercol}{RGB}{5,2,82} % Border color of content boxes
\definecolor{headercol1}{RGB}{5,2,82} % Background color for the header in the content boxes (left side)
\definecolor{headercol2}{RGB}{5,2,82} % Background color for the header in the content boxes (right side)
\definecolor{headerfontcol}{RGB}{255,255,255} % Text color for the header text in the content boxes
\definecolor{boxcolor}{RGB}{255,255,255} % Background color for the content in the content boxes

\begin{document}

\background{ % Set the background to an image (background.pdf)

}

\begin{poster}{
grid=false,
borderColor=bordercol, % Border color of content boxes
headerColorOne=headercol1, % Background color for the header in the content boxes (left side)
headerColorTwo=headercol1, % Background color for the header in the content boxes (right side)
headerFontColor=headerfontcol, % Text color for the header text in the content boxes
boxColorOne=boxcolor, % Background color for the content in the content boxes
headershape=roundedright, % Specify the rounded corner in the content box headers
headerfont=\Large\sf\bf, % Font modifiers for the text in the content box headers
textborder=rectangle,
background=user,
headerborder=open, % Change to closed for a line under the content box headers
boxshade=plain,
columns=4
}
{}
%
%----------------------------------------------------------------------------------------
%	TITLE AND AUTHOR NAME
%----------------------------------------------------------------------------------------
%
%\vspace{2em}
{
%\newline
\sf\bf Temporal properties of social networks} % Poster title
{
\vspace{1em} {\LARGE\bf \underline{Team 18}}\\
%\vspace{1em}
Rik Schreurs, Francois van der Ven, Loek Tonnaer\\ % Author names
{\smaller \{h.c.m.scheurs, f.v.d.ven, l.m.a.tonnaer\}@student.tue.nl}\\
\vspace{1em}
} % Author email addresses
%

%----------------------------------------------------------------------------------------
%	DATA
%----------------------------------------------------------------------------------------

\headerbox{Data}{name=introduction,column=0,row=0, span=2}{
\begin{itemize}
\item Edge data from three social networks, connections between users
\end{itemize}

\begin{center}
\begin{tabular}{l|l|l|l|l}
            & Facebook & YouTube   & Flickr     &  \\ \cline{1-4}
\# edges    & 817.037  & 9.375.374 & 33.140.017 &  \\ \cline{1-4}
\# vertices & 63.731   & 3.223.589 & 2.302.925  &  \\ \cline{1-4}
Size        & 15,5 MB  & 257,1 MB  & 901,5 MB   &  \\ \cline{1-4}
\end{tabular}
\end{center}


\underline{\textbf{Programming setup}}
\begin{itemize}
\item \emph{Apache Flink}
\item Graph processing API and library: \emph{Gelly}
\end{itemize}
}

%\headerbox{True vs False Motifs}{name=introduction2,column=1,row=0, span=2}{
%\includegraphics[scale=0.42]{motif_study}
%}

%----------------------------------------------------------------------------------------
%	OBJECTIVES
%----------------------------------------------------------------------------------------

\headerbox{Objectives}{name=methods,column=2,row=0,span=2}{


\underline{\textbf{Temporal aspects}}
\begin{itemize}
\item How long does it take for a new user to achieve a certain \textbf{number of connections}?
\item \textbf{Degrees of separation}, in how many steps can you go from one user to any other? How does this evolve over time?
\item How \textbf{``connected''} is each of the graphs? How many connected components are there, and how does this change over time?
\end{itemize}


}





%----------------------------------------------------------------------------------------
%	RESULTS
%----------------------------------------------------------------------------------------
\headerbox{Results}{name=results2,span=4,column=0,below=introduction}{
\begin{figure}[H]
\centering
\begin{minipage}[b]{0.45\linewidth}
	\includegraphics[width=1\linewidth]{edges.png}
\end{minipage}
\quad
\begin{minipage}[b]{0.45\linewidth}
	\includegraphics[width=1\linewidth]{std-dev.png}
\end{minipage}

\begin{minipage}[b]{0.3\linewidth}
	\includegraphics[width=1\linewidth]{FacebookConnectedComponents.png}
\end{minipage}
\begin{minipage}[b]{0.3\linewidth}
	\includegraphics[width=1\linewidth]{YouTubeConnectedComponents.png}
\end{minipage}
\begin{minipage}[b]{0.3\linewidth}
	\includegraphics[width=1\linewidth]{FlickrConnectedComponents.png}
\end{minipage}

\end{figure}
}
\headerbox{Results}{name=results,span=2,column=0,below=results2}{
\begin{figure}[H]
	\centering
\begin{minipage}[b]{0.8\linewidth}
	\includegraphics[width=1\linewidth]{FacebookDegreesOfSeperation.png}
\end{minipage}
\end{figure}
}

%----------------------------------------------------------------------------------------
%	CONCLUSIONS
%----------------------------------------------------------------------------------------
\headerbox{Conclusions}{name=conclusion,span=2,column=2,below=results2}{
\begin{itemize}
\item The number of edges per vertex increases for all social networks. In the dataset we have, Flickr keeps increasing at a fast linear rate, Facebook has an exponential rate and Youtubes growth seems to be asymptotic.
\item Degrees of seperation is almost impossible to compute without cloud computing. The degree of seperation of Facebook does behave as expected, as it keeps decreasing.
\item The number of connected components for the Facebook dataset decreases over time. A possible explanation for this is that groups of friends get bigger. YouTube subscribers follow a similar pattern. As channels get bigger, they become central \textbf{hubs} and they connect people to each other. The Flickr dataset does not show interesting characteristics regarding connected components.
\end{itemize}
}

%----------------------------------------------------------------------------------------
\vspace*{20em}
\includegraphics[width=0.3\textwidth]{tue-logo}

\end{poster}
\end{document}